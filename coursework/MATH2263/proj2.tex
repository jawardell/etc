\documentclass[12pt]{article}
\usepackage{tikz}
\usepackage{harpoon}
\usepackage{setspace}
\usepackage{extsizes}
\usepackage{float}
\usepackage{amsmath, amsthm, amssymb}    
\usepackage[margin=.5in]{geometry}
\usepackage{pgfplots}


\title{\vspace{-2.0cm}Project 2}
\author{Joanne Wardell}
\date{Tuesday, October 16, 2018}
\begin{document}
\maketitle

\subsection*{Part I }
\noindent $f(x,y) = 2y^{2} -4xy+2x+5$, \hspace{10pt} $0 \leq x \leq 1$, \hspace{10pt} $0 \leq y \leq 2$\\
\noindent 1.)


\begin{tikzpicture}
  \begin{axis}[
      xmin=0,xmax=3,
      ymin=-1,ymax=8,
      grid=both,
      grid style={line width=.1pt, draw=gray!10},
      major grid style={line width=.2pt,draw=gray!50},
      axis lines=middle,
      minor tick num=5,
      enlargelimits={abs=0.5},
      axis line style={latex-latex},
      ticklabel style={font=\tiny,fill=white},
      xlabel style={at={(ticklabel* cs:1)},anchor=north west},
      ylabel style={at={(ticklabel* cs:1)},anchor=south west},
      restrict x to domain=-1:inf
  ]
      \addplot +[mark=none, ultra thick] coordinates {(1, -1) (1, 8)};
      \addplot +[mark=none, ultra thick] coordinates {(0, -1) (0, 8)};
	  \addplot[mark=none, ultra thick]{2};
	  \addplot[mark=none, ultra thick]{0};

  \node [color=black] at (140,360) {$y=2$};
  \node [color=black] at (80,170) {$x=0$};
  \node [color=black] at (180,170) {$x=1$};
  \node [color=black] at (140,60) {$y=0$};





\end{axis}
\end{tikzpicture}




\noindent 2.)The critical points of $f$ that are interior to the domain are $(\frac{1}{2}, \frac{1}{2})$.\\\\
\noindent $\frac{\partial f}{\partial x} = -4y + 2$\\
\noindent $-4y + 2 = 0 \rightarrow y = \frac{1}{2}$\\
\noindent $\frac{\partial f}{\partial y} = 4y -4x$\\
\noindent $4y - 4x = 0 \\ 2x - 4x = 0 \rightarrow x = \frac{1}{2}$\\

\noindent 3.) The critical points along the edges of the domain are $(0,0)$ and $(1,1)$\\\\
\noindent $f(0, y) = 2y^{2} + 5$\\
\noindent $\frac{df}{dy} = 4y$, \hspace{10pt} $4y = 0 \rightarrow y = 0$\\
\noindent $\mathbf{(0, 0)}$\\\\

\noindent $f(x, 0) = 2x + 5$\\
\noindent $\frac{df}{dx} = 2$\\
\noindent Since $\frac{df}{dx}$ with this edge resitriction is never zero,
the function does not have a critical point along this edge.\\\\

\noindent $f(x, 2) = -6x + 13$\\
\noindent $\frac{df}{dx} = -6$\\
\noindent Since $\frac{df}{dx}$ with this edge resitriction is never zero,
the function does not have a critical point along this edge.\clearpage

\noindent $f(1, y) = 2y^{2} -4y + 7$\\
\noindent $\frac{df}{dy} = 4y - 4$, \hspace{10pt} $4y - 4 = 0 \rightarrow y = 1$\\
\noindent $\mathbf{(1, 1)}$\\\\


\noindent 4.) \\\\


\begin{table}[h!]
  \begin{center}
    \label{tab:table1}
    \begin{tabular}{l|c|r} % <-- Alignments: 1st column left, 2nd middle and 3rd right, with vertical lines in between
      $(x,y)$ & $f(x,y)$ &  \\
      \hline
      $(0, 0)$ & 5 & \\
      $(1,1)$ & 8  & \\
      $(\frac{1}{2}, \frac{1}{2})$ & $\frac{11}{2}$ & \\
      $(2,0)$ & 9  & maximum\\
      $(1,2)$ & 1  & minimum \\
      $(1,0)$ & 7  & \\
    \end{tabular}
  \end{center}
\end{table}

\noindent Within its domain restriction,  $f$ has an absolute min at $f(1, 0) = 7$ and an absolute max at $f(2, 0) = 9$.\\\\




\subsection*{Part II} 
\noindent 1.) The area of the triangle $T$ can be expressed as half the magnitude of the 
cross product between two vectors.\\\\
\noindent Let vector $\overrightharp{PA} = \langle a, 0, 0 \rangle - \langle 0, 0, 1\rangle = \langle a, 0, -1\rangle$ and vector $\overrightharp{PB} = \langle 0, b, 0 \rangle - \langle 0, 0, 1 \rangle = \langle 0, b, -1\rangle$. \\\\
\noindent Let $A$ be the area of $T$. Then $A = \frac{1}{2} \| \overrightharp{PA} \times \overrightharp{PB} \| \\\\
=\frac{1}{2}\|(0)(-1)\mathbf{i} + (0)(-1)\mathbf{j} + (a)(b)\mathbf{k} - (-1)(b)\mathbf{i} - (-1)(a)\mathbf{j} - (0)(0)\mathbf{k}\| 
\\\\=  \frac{1}{2}\| \langle b , a , ab\rangle \| 
\\\\ = \frac{1}{2}\sqrt{a^{2} + b^{2} + (ab)^{2}}$.\\\\


\noindent 2.) The goal is to find the extreme points of $A$ with restrictions $a+b = 12$, \hspace{10pt} $a,b \geq 0$\\\\
\noindent Maximize/Minimize: $f = \frac{1}{2}\sqrt{a^{2} + b^{2} + (ab)^{2}}$\\
\noindent Constraint: $g = a + b$\\\\
\noindent The extreme points shared by the two functions are points that exist on the functions simultaneously. Thus, 
we are seeking the points at which the two functions are touching tangentially. If the fucntions are touching at some point, then $\nabla f \propto \nabla g$ at the point.\\
\noindent Use $\lambda$ as the proportionality constant to set up an equation involving the gradients.\\\\
\noindent $\nabla f = \lambda \nabla g$\\\\
\noindent $\langle \frac{a + ab^{2}}{2\sqrt{a^{2} + b^{2} + a^{2}b^{2}}},\frac{b + ba^{2}}{2\sqrt{a^{2} + b^{2} + a^{2}b^{2}}} \rangle = \lambda \langle 1, 1\rangle$\\\\
\noindent Equate coefficients and use the constraint equation: \\\\
\noindent $\frac{a + ab^{2}}{2\sqrt{a^{2} + b^{2} + a^{2}b^{2}}} = \lambda$\\\\
\noindent $\frac{b + ba^{2}}{2\sqrt{a^{2} + b^{2} + a^{2}b^{2}}} = \lambda$\\\\
\noindent $a + b = 12 \rightarrow a = 12 - b$\\\\
\noindent $\frac{a + ab^{2}}{2\sqrt{a^{2} + b^{2} + a^{2}b^{2}}} =\frac{b + ba^{2}}{2\sqrt{a^{2} + b^{2} + a^{2}b^{2}}}$\\\\
\noindent $\frac{(12-b) + (12-b)^{2}}{2\sqrt{(12-b)^{2} + b^{2} + (12-b)^{2}b^{2}}} =\frac{b + b(12-b)^{2}}{2\sqrt{(12-b)^{2} + b^{2} + (12-b)^{2}b^{2}}}$\\\\
\noindent $\frac{12-b + b^{2}-24b + 144}{2\sqrt{b^{2}-24b+144 + b^{2} + b^{4}-24b^{3}+144b^{2}}} =\frac{b + b^{3} -24b^{2} + 144b}{2\sqrt{b^{2}-24b + 144 + b^{2} + b^{4}-24b^{3}+144b^{2}}}$\\\\
\noindent $\frac{b^{2}-25b + 156}{2\sqrt{b^{4} -24b^{3} + 146b^{2}-24b+144}} =\frac{b^{3} -24b^{2} + 145b}{2\sqrt{b^{4} -24b^{3} + 146b^{2}-24b + 144 }}$\\\\
\noindent $b^{2}-25b + 156 = b^{3} - 24b^{2}+145b$\\\\ 
\noindent $-b^{3} + 25b^{2} -160b= -156$\\\\ 
\noindent $b^{3} - 25b^{2} + 160b= 156$\\\\ 
\noindent $b(b^{2} - 25b + 160)= 156$\\\\\\
\noindent Using a TI-84 graphing calculator, solve for $b$ using the Newton Raphson method.\\\\
\noindent $b \approx 1.18349$, \hspace{10pt} $b \approx 8.74706$, $b \approx 15.06945$\\
\noindent If $b = 1.18349$, then $a = 12 -1.18349 \approx 10.81651$.\\
\noindent If $b = 8.74706$, then $a = 12 - 8.74706 \approx 3.25294$.\\
\noindent If $b = 15.06945$, then $a < 0$. This case is not usable because it defies the constraint.\\\\
\noindent $f(10.81651, 1.18349) \approx 8.40043$\\
\noindent $f(3.25294, 15.06945) \approx 25.69354$\\\\
\noindent The area is maximized when  $a = 3.25294 \text{ and } b = 15.06945$.  \\The area is minimized when $a = 10.81651 \text{ and } b = 1.18349$.
 

\end{document}
