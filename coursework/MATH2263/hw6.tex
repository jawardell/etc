\documentclass[12pt]{article}
\usepackage{tikz}
\usepackage{harpoon}% <---
\usepackage{setspace}
\usepackage{extsizes}
\usepackage{float}
\usepackage{amsmath, amsthm, amssymb}    
\usepackage[margin=.5in]{geometry}


\title{\vspace{-2.0cm}Homework 6}
\author{Joanne Wardell}
\date{Thursday, September 13, 2018}
\begin{document}
\maketitle


\subsection*{Section 13.4}
\noindent 4.) $\vec{r}(t) = (cos{t} + tsin{t})\mathbf{i} + (sin{t} - tcos{t})\mathbf{j}$, \hspace{10pt} $t > 0$\\\\
\noindent Find the tangent velocity vector by taking the first derivative of $\vec{r}$ with respect to time.\\\\
\noindent $\frac{d\vec{r}}{dt} = \frac{d(cos{t} + tsin{t})}{dt}\mathbf{i} + \frac{d(sin{t} - tcos{t})}{dt}\mathbf{j} + \frac{d0}{dt}\mathbf{k}= tcos(t)\mathbf{i} + tsin(t)\mathbf{j}$\\\\\\
\noindent Normalize the velocity vector for $\vec{T}$. \\\\
\noindent $\vec{T} = \langle cos{t}, sin{t}, 0\rangle$\\\\
\noindent Use the definition of the priniciple unit normal to calculate $\vec{N}$.\\\\
\noindent $\vec{N} = \frac{1}{\kappa}\frac{d\vec{T}}{dt} = \frac{1}{\kappa}\langle \frac{dcos{t}}{dt}, \frac{dsin{t}}{dt} , \frac{d0}{dt}\rangle = \langle -sin{t}, cos{t}, 0\rangle$\\\\
\noindent $\kappa$ is $\frac{1}{\| \vec{v} \|}\frac{d\vec{T}}{dt}$ which is $ \frac{1}{t} \|\langle -sin{t} + cos{t} \rangle \| = \frac{1}{t}$\\\\\\
\noindent 12.) $\vec{v} =\vec{r}(t) = (6sin(2t))\mathbf{i} + (6cos(2t))\mathbf{j} + 5t\mathbf{k}$\\\\
\noindent $\frac{d\vec{r}}{dt} = 6\frac{dsin{2t}}{dt}\mathbf{i} + 6\frac{dcos{2t}}{dt}\mathbf{j} + 5\frac{dt}{dt}\mathbf{k} =
12cos{2t}\mathbf{i} - 12sin{2t}\mathbf{j} + 5\mathbf{k}$\\\\
\noindent $\| \vec{v} \| = \sqrt{144 + 25} = 13 $\\\\
\noindent $\vec{T} = \frac{\vec{v}}{\| \vec{v} \|} = \langle \frac{12cos{2t}}{13}, -\frac{12sin{2t}}{13}, \frac{5}{13}\rangle$\\\\
\noindent $\frac{d\vec{T}}{dt} = \langle \frac{12}{13}\frac{dcos{2t}}{dt} , -\frac{12}{13}\frac{dsin{2t}}{dt} + \frac{5}{13}\frac{d}{dt}\rangle
= \langle -\frac{24sin{2t}}{13}, -\frac{24cos{2t}}{13}, 0 \rangle$\\\\
\noindent $\| \frac{d\vec{T}}{dt} \|  = \frac{24}{13}$\\\\
\noindent $\kappa = \frac{1}{\| \vec{v} \| }\frac{d\vec{T}}{dt} = \frac{1}{13}\frac{24}{13} =\frac{24}{169}$\\\\
\noindent $\vec{N} = \frac{1}{\kappa}\frac{d\vec{T}}{dt} = \frac{169}{24}\langle -\frac{24sin{2t}}{13}, -\frac{24cos{2t}}{13}, 0\rangle 
 = \langle -13sin{2t}, -13cos{2t}, 0\rangle $\\\\
\noindent 30.) The parameterization of the osculating circle for the parabola $y = x^{2}$ when $x = 1$ is found 
by using the equation of the circle $\vec{r}(t) = \langle Rcos(t) + h, Rsin(t) + k \rangle$. First, obtain the circle-of-best-fit's radius by finding $\kappa$.\\\\
\noindent $\vec{T} = \frac{}{}$\\\\
\noindent $\vec{N} = \frac{}{}$\\\\
\noindent $a_{T} = $\\\\
\noindent $a_{N} = $\\\\
\noindent $\vec{a} = a_{T}\vec{T} + a_{N}\vec{N} = \langle \rangle$\\\\
\noindent $\vec{T} \times \vec{N}  = \langle \rangle$\\\\
\noindent $\| \vec{T} \times \vec{N} \|  = \sqrt{} = $\\\\
\noindent $\kappa = \frac{\| \vec{v} \times \vec{a} \|}{\| \vec{v} \| ^{2}} = $\\\\
\noindent $R = \kappa^{-1} = $\\\\
\noindent $R_{x = 1} =  = $\\\\


 \clearpage



\subsection*{Section 13.5}
\noindent 2.)$\vec{r}(t) = (1 + 3t)\mathbf{i} + (t - 2)\mathbf{j} - 3t\mathbf{k}$\\\\
\noindent $\vec{v} = \langle 3, 1, 3\rangle$, \hspace{10pt} $\| \vec{v} \| = \sqrt{19}$\\\\
\noindent $a_{T} = \frac{d\|\vec{v} \|}{dt} = \frac{d0}{dt} = 0$\\\\
\noindent $a_{N} = \kappa \| \vec{v} \|^{2} = \kappa 0^{2} = 0$\\\\
\noindent $\vec{a} = a_{T}\vec{T} + a_{N}\vec{N} = 0\vec{T} + 0\vec{N}$\\\\

\noindent 4.) $\vec{r}(t) = (tcos{t})\mathbf{i} + (tsin{t})\mathbf{j} + t^{2}\mathbf{k}$, \hspace{10pt} $t = 0$\\\\
\noindent $\vec{v} = \langle cos{t} - tsin{t}, sin{t} + tcos{t}\rangle$, \hspace{10pt} $\| v\| = \sqrt{1+5t^{2}}$\\\\
\noindent $a_{T} = \frac{d\|\vec{v} \| }{dt} = \frac{5t}{\sqrt{1+5t^{2}}}$, \hspace{10pt} $a_{T}\Big|_{0} = 0$\\\\
\noindent $\vec{a} = \frac{d\vec{v}}{dt} = \langle 0, 2, 2\rangle$, \hspace{10pt} $\| \vec{a}\|  =2\sqrt{2}$\\\\
\noindent $a_{N} = \sqrt{\| a \| ^{2} - (a_{T})^{2}} = \sqrt{8 - 0} = 2\sqrt{2}$\\\\
\noindent $\vec{a} = 0\vec{T} + 2\sqrt{2}\vec{N}$\\\\


\noindent 8.) $\vec{r}(t) = (cos{t})\mathbf{i} + (sin{t})\mathbf{j} + t\mathbf{k}$, \hspace{10pt} $t = 0$\\\\
\noindent $\frac{d\vec{r}}{dt} = \langle -sin{t}, cos{t}, 1\rangle$,
 \hspace{10pt} $\langle -sin{t}, cos{t}, 1\rangle \Big|_{0} = \langle 1, 0, 1\rangle$\\\\
\noindent $\| \vec{v} \| = \langle \sqrt{2}\rangle$, \hspace{10pt} $\| v \| \Big|_{0} = \langle \sqrt{2} \rangle$\\\\
\noindent $\vec{T} = \langle -\frac{sin{t}}{\sqrt{2}}, \frac{cos{t}}{\sqrt{2}}, \frac{1}{\sqrt{2}}\rangle$,
 \hspace{10pt} $\vec{T}\Big|_{0} = \langle \frac{1}{\sqrt{2}}, 0, \frac{1}{\sqrt{2}}\rangle$\\\\
\noindent $\frac{d\vec{T}}{dt} = \langle -\frac{cos{t}}{\sqrt{2}}, -\frac{sin{t}}{\sqrt{2}}, 0 \rangle$\\\\
\noindent $\vec{N} = \hat{\frac{d\vec{T}}{dt}} = \langle -cos{t}, -sin{t}\rangle$, \hspace{10pt}
 $\vec{N}\Big|_{0} = \langle -1, 0, 0\rangle$\\\\
\noindent $\vec{B} = \vec{T} \times \vec{N} = \langle 0, -\frac{1}{\sqrt{2}}, \frac{1}{\sqrt{2}}\rangle$\\\\
\noindent The equation for the \textbf{osculating plane} at $t = 0$ is \\$Ax + Bx + Cx  = d$...
$0(x - 1) - \frac{1}{\sqrt{2}}(y - 0) + \frac{1}{\sqrt{2}} (z - 0) = 0$\\\\
\noindent $\Rightarrow -y+z = 0$ or $y - z = 0$\\\\ 
\noindent The equation for the \textbf{normal plane} at $t = 0$ is \\$ Ax + Bx + Cx  = d$...
$0(x - 1)  \frac{1}{\sqrt{2}}(y - 0) + \frac{1}{\sqrt{2}} (z - 0) = 0$\\\\
\noindent $\Rightarrow y+z = 0$\\\\ 
\noindent The equation for the \textbf{rectifying plane} at $t = 0$ is \\$ Ax + Bx + Cx  = d$...
$-(x - 1) + 0(y - 0) + 0(z - 0) = 0$\\\\
\noindent $\Rightarrow x = 1$\\\\ 


\noindent 12.) $\vec{r}(t) = (6sin{2t})\mathbf{i} + (6cos{2t})\mathbf{j} + 5t\mathbf{k}$\\\\
\noindent $\frac{d\vec{r}}{dt} = 6\frac{dsin{2t}}{dt}\mathbf{i} + 6\frac{dcos{2t}}{dt}\mathbf{j} + 5\frac{dt}{dt}\mathbf{k} =
12cos{2t}\mathbf{i} - 12sin{2t}\mathbf{j} + 5\mathbf{k}$\\\\
\noindent $\| \vec{v} \| = \sqrt{144 + 25} = 13 $\\\\
\noindent $\vec{T} = \frac{\vec{v}}{\| \vec{v} \|} = \langle \frac{12cos{2t}}{13}, -\frac{12sin{2t}}{13}, \frac{5}{13}\rangle$\\\\
\noindent $\frac{d\vec{T}}{dt} = \langle \frac{12}{13}\frac{dcos{2t}}{dt} , -\frac{12}{13}\frac{dsin{2t}}{dt} + \frac{5}{13}\frac{d}{dt}\rangle
= \langle -\frac{24sin{2t}}{13}, -\frac{24cos{2t}}{13}, 0 \rangle$\\\\
\noindent $\| \frac{d\vec{T}}{dt} \|  = \frac{24}{13}$\\\\
\noindent $\kappa = \frac{1}{\| \vec{v} \| }\frac{d\vec{T}}{dt} = \frac{1}{13}\frac{24}{13} =\frac{24}{169}$\\\\
\noindent $\vec{N} = \frac{1}{\kappa}\frac{d\vec{T}}{dt} = \frac{169}{24}\langle -\frac{24sin{2t}}{13}, -\frac{24cos{2t}}{13}, 0\rangle $\\\\
\noindent $\vec{B} = \vec{T} \times \vec{N} = \langle \frac{5}{13}cos{2t}, -\frac{5}{13}sin{2t}, -\frac{12}{13}\rangle$\\\\
\noindent $\tau = -\frac{1}{\| \vec{v} \| }(\frac{d\vec{B}}{dt}\cdot \vec{N}) = 13(\langle -\frac{10sin{2t}}{13}, -\frac{10cos{2t}}{13}, 0\rangle \cdot \langle -\frac{24sin{2t}}{13}, -\frac{24cos{2t}}{13}, 0\rangle) = 13\frac{240}{169} = \frac{240}{13}$ wrong\\\\
\noindent 24.) The following can be said about the torsion of a smooth plane curve $\vec{r}(t) = f(t)\mathbf{i} + g(t)\mathbf{j}$, \hspace{10pt} $t > 0$\\\\
\noindent 26.) Using the formula $\tau = -\frac{1}{\| \vec{v} \|}(\frac{d\vec{B}}{dt} \cdot \vec{N})$, the torsion of the helix in Example 2 is:\\\\
\noindent $\vec{v} = $\\\\
\noindent $\vec{T} = $\\\\
\noindent $\vec{N} = $\\\\
\noindent $\vec{B} = \vec{T} \times \vec{N} = $\\\\
\noindent $\frac{d\vec{B}}{dt} = \langle \rangle$\\\\
\noindent $\tau = -\frac{1}{\| \vec{v} \|}(\frac{d\vec{B}}{dt} \cdot \vec{N}) = -\frac{1}{\| \vec{v} \| } \frac{}{}$

\end{document}
