\documentclass[12pt]{article}
\usepackage{tikz}
\usepackage{harpoon}% <---
\usepackage{setspace}
\usepackage{extsizes}
\usepackage{float}
\usepackage{amsmath, amsthm, amssymb}    
\usepackage[margin=.5in]{geometry}


\title{\vspace{-2.0cm}Homework 6}
\author{Joanne Wardell}
\date{Thursday, September 13, 2018}
\begin{document}
\maketitle


\subsection*{Section 13.4}
\noindent 4.) $\vec{r}(t) = (cos{t} + tsin{t})\mathbf{i} + (sin{t} - tcos{t})\mathbf{j}$, \hspace{10pt} $t > 0$\\\\
\noindent Find the tangent velocity vector by taking the first derivative of $\vec{r}$ with respect to time.\\\\
\noindent $\frac{d\vec{r}}{dt} = \frac{d(cos{t} + tsin{t})}{dt}\mathbf{i} + \frac{d(sin{t} - tcos{t})}{dt}\mathbf{j} + \frac{d0}{dt}\mathbf{k}= tcos(t)\mathbf{i} + tsin(t)\mathbf{j}$\\\\\\
\noindent Normalize the velocity vector for $\vec{T}$. \\\\
\noindent $\vec{T} = \langle cos{t}, sin{t}, 0\rangle$\\\\
\noindent Use the definition of the priniciple unit normal to calculate $\vec{N}$.\\\\
\noindent $\vec{N} = \frac{1}{\kappa}\frac{d\vec{T}}{dt} = \frac{1}{\kappa}\langle \frac{dcos{t}}{dt}, \frac{dsin{t}}{dt} , \frac{d0}{dt}\rangle = \langle -sin{t}, cos{t}, 0\rangle$\\\\
\noindent $\kappa$ is the inverse of the  magnitude of $\frac{d\vec{T}}{dt}$ which is $ (\sqrt{(-sin{t})^{2} + (cos{t})^{2}})^{-1} = 1$\\\\
\noindent 12.) $\vec{v} =\vec{r}(t) = (6sin(2t))\mathbf{i} + (6cos(2t))\mathbf{j} + 5t\mathbf{k}$\\\\
\noindent $\frac{d\vec{r}}{dt} = 6\frac{dsin{2t}}{dt}\mathbf{i} + 6\frac{dcos{2t}}{dt}\mathbf{j} + 5\frac{dt}{dt}\mathbf{k} =
12cos{2t}\mathbf{i} - 12sin{2t}\mathbf{j} + 5\mathbf{k}$\\\\
\noindent $\| \vec{v} \| = \sqrt{144 + 25} = 13 $\\\\
\noindent $\vec{T} = \frac{\vec{v}}{\| \vec{v} \|} = \langle \frac{12cos{2t}}{13}, -\frac{12sin{2t}}{13}, \frac{5}{13}\rangle$\\\\
\noindent $\frac{d\vec{T}}{dt} = \langle \frac{12}{13}\frac{dcos{2t}}{dt} , -\frac{12}{13}\frac{dsin{2t}}{dt} + \frac{5}{13}\frac{d}{dt}\rangle
= \langle -\frac{24sin{2t}}{13}, -\frac{24cos{2t}}{13}, 0 \rangle$\\\\
\noindent $\| \frac{d\vec{T}}{dt} \|  = \frac{24}{13}$\\\\
\noindent $\kappa = \frac{1}{\| \vec{v} \| }\frac{d\vec{T}}{dt} = \frac{1}{13}\frac{24}{13} =\frac{24}{169}$\\\\
\noindent $\vec{N} = \frac{1}{\kappa}\frac{d\vec{T}}{dt} = \frac{169}{24}\langle -\frac{24sin{2t}}{13}, -\frac{24cos{2t}}{13}, 0\rangle 
 = \langle -13sin{2t}, -13cos{2t}, 0\rangle $\\\\
\noindent 30.) The parameterization of the osculating circle for the parabola $y = x^{2}$ when $x = 1$ is \clearpage



\subsection*{Section 13.5}
\noindent 2.)$\vec{r}(t) = (1 + 3t)\mathbf{i} + (t - 2)\mathbf{j} - 3t\mathbf{k}$\\\\
\noindent 4.) $\vec{r}(t) = (tcos{t})\mathbf{i} + (tsin{t})\mathbf{j} + t^{2}\mathbf{k}$, \hspace{10pt} $t = 0$\\\\
\noindent 8.) $\vec{r}(t) = (cos{t})\mathbf{i} + (sin{t})\mathbf{j} + t\mathbf{k}$, \hspace{10pt} $t = 0$\\\\
\noindent 12.) $\vec{r}(t) = (6sin{2t})\mathbf{i} + (6cos{2t})\mathbf{j} + 5t\mathbf{k}$\\\\
\noindent 24.) The following can be said about the torsion of a smooth plane curve $\vec{r}(t) = f(t)\mathbf{i} + g(t)\mathbf{j}$, \hspace{10pt} $t > 0$\\\\
\noindent 26.) Using the formula $\tau = -\frac{1}{\| \vec{v} \|}(\frac{d\vec{T}}{dt} \cdot \vec{N})$, the torsion of the helix in Example 2 is:\\\\

\end{document}
