\documentclass[12pt]{article}
\usepackage{pgfplots}
\usepackage{harpoon}% <---
\usepackage{setspace}
\usepackage{extsizes}
\usepackage{pgfplots}
\usepackage{float}

\usepackage{tikz-3dplot}
\usepackage{tikz}
\usetikzlibrary{quotes,angles,positioning}

\usepackage{amsmath, amsthm, amssymb}
\usepackage[margin=.5in]{geometry}
\usepackage{xfrac}
\usetikzlibrary{calc}
\usepgfplotslibrary{fillbetween}
\usepgfplotslibrary{polar}
\pgfplotsset{compat=1.11}

\title{\vspace{-2.0cm}Optics Problem Set 1}
\author{Joanne Wardell}
\date{Tuesday, February 12, 2019}
\begin{document}
\maketitle

\subsection*{Chapter 2}

\noindent  2.40) Given the traveling wave $\Psi(x,t) = 5.0$exp$(-ax^{2}-bt^{2}-
2\sqrt{ab}xt)$, determine its direction of propagation. Calculate a few 
values of $\Psi$ and make a sketch of the wave at $t = 0$, taking $a = 25m^{-2}$ and 
b = $b = 9.0s^{-2}$. What is the speed of the wave?\\\\
\noindent After factoring notice that 
$\Psi = 5.0$exp$([-(x+\sqrt{\frac{b}{a}}t^{2})^{2}])$. Thus, the direction 
of propagation is in the \textbf{negative x direction}.\\\\
The following is $\Psi$ as $x$ varies.\\
\begin{tabular}{c|c}
  $\mathbf{x}$ & $\mathbf{\Psi}$ \\ \hline
  0.6 & 0.0006\\
  0.4 & 0.09\\
  0.2 & 1.8\\
  0.0 & 5.0\\
  -0.2 & 1.8\\
  -0.4 & 0.09\\
  -0.6 & 0.0006\\
\end{tabular} \\\\

\noindent Let $t = 0$, $a = 25m^{-2}$, and $b = 3s^{2}$. Then 
$\Psi = -5.0$exp$25x^{2}$.\\\\
\noindent Below is a graph of $\Psi$ at time $t = 0$.\\\\


\noindent $\Psi$ is a function of $(x + vt)$, where $x + vt = (x + \sqrt{\frac{a}{b}}t)$. 
Thus, the speed of the wave is $v = \sqrt{\frac{a}{b}} = \sqrt{\frac{9m^{2}}{25s^{2}}} 
= \frac{3m}{5s} = 0.6ms^{-1}$.\\\\

\noindent  2.44) Working with exponentials directly, show that the magnitude of 
$\Psi = Ae^{i\omega t}$ is $A$. Then, rederive the same result using Euler's formula. 
Prove that $e^{i\alpha}e^{i\beta} = e^{(\alpha + \beta)}$\\\\

\noindent Let $\tilde{z} = A^{i\omega t}$
\noindent and its congugate $\tilde{z}^{*} = Ae^{-i\omega t}$.
\noindent Let the modulus, or magnitude, of $\tilde{z}$ 
be $\| \tilde{z}\| = (\tilde{z}\tilde{z}^{*})^{\frac{1}{2}}$.

\noindent Thus, the magnitude of $\Psi$ is 
$\| \tilde{z}\| =  (\tilde{z}\tilde{z}^{*})^{\frac{1}{2}} 
= [(Ae^{i\omega t})(Ae^{-i\omega t})]^{\frac{1}{2}}
= [A^{2}e^{i\omega t-i\omega t}]^{\frac{1}{2}}
=\sqrt{A^{2}} = A$.\\\\

\noindent Euler's formula states that $e^{i\omega t} = \cos{\omega t} + i\sin{\omega t}$.
Notice that $\omega = 2\pi\nu = \frac{2\pi}{t}$. Plugging this in yeilds $Ae^{i\frac{2\pi}{\tau}t} = A[\cos{\frac{2\pi}{\tau}t} + i\sin{\frac{2\pi}{\tau}t}]$.\\
If $t = \tau$, then $Ae^{i\frac{2\pi}{t}t} = A[\cos{\frac{2\pi}{t}t} + i\sin{\frac{2\pi}{t}t}]
=Ae^{2\pi} = A[\cos{2\pi} + i\sin{2\pi}] = A[1 + 0] = A$.\\

\noindent By law of exponents, $e^{x} + e^{y} = e^{x+y}$.\\
Thus, $e^{i\alpha}e^{i\beta} = e^{i\alpha + i\beta} = e^{i(\alpha + \beta)}$.\\\\


\noindent  2.49) Show that $\Psi(x,y,z,t) = f(\alpha x + \beta y + \gamma z - vt)$
and $\Psi(x,y,z,t) = g(\alpha x + \beta y + \gamma z - vt)$ which are plane waves of arbitrary form 
satisfy the three-dimensional wave equation.\\\\

\noindent Since $f$ and $g$ are functions which are twice differentiable, compute the partial 
derivatives of $\Psi$\\

\noindent $\frac{\partial^{2}\Psi}{\partial x^{2}} = -\alpha^{2}k^{2}\Psi$\hspace{50pt}
$\frac{\partial^{2}\Psi}{\partial y^{2}} = -\beta^{2}k^{2}\Psi$\hspace{50pt}
$\frac{\partial^{2}\Psi}{\partial z^{2}} = -\gamma^{2}k^{2}\Psi$\hspace{50pt}
$\frac{\partial^{2}\Psi}{\partial t^{2}} = -k^{2}v^{2}\Psi$\hspace{50pt}\\\\

\noindent Add the three spatial derivatives
$\frac{\partial^{2} \Psi}{\partial x^{2}} + \frac{\partial^{2}\Psi}{\partial y^{2} }
+\frac{\partial ^{2} \Psi}{\partial z^{2}} = -k^{2}(\alpha + \beta + \gamma)\Psi$
and recall that $\alpha + \beta + \gamma = 1$. Thus
$-k^{2}(\alpha + \beta + \gamma)\Psi = -k^{2}\Psi$.\\\\

\noindent Plugging the second partial of $\Psi$ with respect to $t$ 
into the 3-d wave function yeilds 
$\frac{1}{v^{2}}\frac{\partial ^{2} \Psi}{\partial t^{2}} = -k^{2}\Psi$.\\\\


\noindent  2.46) Take the complex quantities $\tilde{z}_{1} = (x_{1}+iy_{1})$
$\tilde{z}_{2} = (x_{2}+iy_{2})$ and show that \\ 
\[\mathbf{Re}(\tilde{z}_{1} + \tilde{z}_{2}) 
= \mathbf{Re}(\tilde{z}_{1}) + \mathbf{Re}(\tilde{z}_{2})\]

\noindent If $\mathbf{Re}$ means to extract the real components of a complex number, then 
since the argument $\tilde{z}_{1} + \tilde{z}_{2}$ resolves to 
$x_{1} + iy_{1} + x_{2} + iy_{2}$, another complex number, simply extract the real component. Thus, the real component from $x_{1} + x_{2} + (y_{1} + y_{2})i$ is 
$x_{1} + x_{2}$. This result is the same as $\mathbf{Re}(\tilde{z}_{1}) + \mathbf{Re}(\tilde{z}_{2}) = \mathbf{Re}(x_{1} + iy_{1}) + \mathbf{Re}(x_{2} + iy_{2})$.\\\\



\noindent  2.47) Take the complex quantities $\vec{z}_{1} = (x_{1} + iy_{1})$
and $\tilde{z}_{2} = (x_{2}+iy_{2})$ and show that 
$\mathbf{Re}(\tilde{z}_{1}) \times \mathbf{Re}(\tilde{z}_{2})
\neq \mathbf{Re}(\tilde{z}_{1}) \times \mathbf{Re}(\tilde{z}_{2})$.\\\\

\noindent Start with the right hand side and notice that 
$\\\mathbf{Re(\tilde{z}_{1})} \times \mathbf{Re(\tilde{z}_{2})}
\\=\mathbf{Re(x_{1} + iy_{1})} \times \mathbf{Re(x_{2} + iy_{2})}
\\=x_{1} \times x_{2} = x_{1}x_{2}$.\\

\noindent Then the lefthand side resolves to 
$\\ \mathbf{Re}(\tilde{z}_{1} \times \tilde{z}_{2})
\\= \mathbf{Re}((x_{1} + iy_{1})\times(x_{2} + iy_{2}))
\\= \mathbf{Re}(x_{1}x_{2} + x_{1}iy_{2} + x_{2}iy_{1} + i^{2}y_{1}y_{2})
\\= \mathbf{Re}(x_{1}x_{2} - y_{1}y_{2} + i(x_{1}y_{2} + x_{2}y_{1}))
\\= x_{1}x_{2}-y_{1}y_{2} \neq x_{1}x_{2}$. \\\\



\noindent  2.56) Show explicitly that the function 
$\Psi(\vec{r}, t) = A$exp$[i(\vec{k}\cdot\vec{r} + \omega t + \epsilon)]$
describes a wave provided that $v = \frac{\omega}{k}$.\\\\

\noindent In order for $\Psi$ to describe a wave, it needs to 
hold true that $v = \frac{\omega}{k}$. Let's assume that $\Psi$ 
is a wave function. Utilize the 3-D wave equation. \\
\noindent $\nabla^{2}\Psi-\frac{1}{v^{2}}\frac{\partial^{2}\Psi}{\partial t} = 0
\\\\\nabla^{2}[A$exp$[i[\vec{k}\cdot\vec{r}+\omega t + \epsilon]]]
-\frac{1}{v^{2}}\frac{\partial^{2}}{\partial t}[A$exp$[i[\vec{k}\cdot\vec{r}+\omega t + \epsilon]]] = 0
\\\\\frac{\partial}{\partial r}[\frac{\partial}{\partial r}[A$exp$[i[\vec{k}\cdot\vec{r}+\omega t + \epsilon]]]]
-\frac{1}{v^{2}}[\frac{\partial}{\partial t}[\frac{\partial}{\partial t}[A$exp$[i[\vec{k}\cdot\vec{r}+\omega t + \epsilon]]]]] = 0
\\\\\frac{\partial}{\partial r}[ik[A$exp$[i[\vec{k}\cdot\vec{r} + \omega t + \epsilon]]]]
- \frac{1}{v^{2}}[\frac{\partial}{\partial t}[i\omega[A$exp$[i[\vec{k}\cdot\vec{r} + \omega t + \epsilon]]]]] = 0
\\\\(ik)^{2}[A$exp$[i[\vec{k}\cdot\vec{r}+\omega t + \epsilon]]]
- \frac{1}{v^{2}}[(i\omega)^{2}[A$exp$[i[\vec{k}\cdot\vec{r} + \omega t + \epsilon]]]] = 0
\\\\-k^{2}[A$exp$[i[\vec{k}\cdot\vec{r} + \omega t + \epsilon]]]
+ \frac{1}{v^{2}}[\omega^{2}[A$exp$[i[\vec{k}\cdot\vec{r}+\omega t + \epsilon]]]] = 0
\\\\-k^{2} = -\frac{1}{v^{2}}(\omega^{2})
\\\\k^{2} = \frac{\omega^{2}}{v^{2}}
\\\\v = \frac{\omega}{k}$



\subsection*{Chapter 3}
\noindent  3.19) A 1.0 mW laser produces a nearly parallel beam $1.0cm^{2}$ 
in cross-sectional area at a wavelength of 650nm. Determine the amplitude 
of the electric field in the beam, assuming the wavefronts are homogeneous and 
the light travels in a vacuum. \\\\
\noindent $P = 1.0\times 10^{-3}W$\\\\
\noindent $A = 1.0\times 10^{-4}m^{2}$\\\\
\noindent $\lambda = 650\times 10^{-9}m$\\\\
\noindent $P = \frac{E_{max}^{2}}{2\mu_{0}c}\rightarrow
E = \sqrt{\frac{2\mu_{0}cP}{A}} = 2.30\times10^{-1}Vm^{-1}$\\\\

\noindent  3.21) The following is the expression for the $\vec{E}$-feild of an 
electromagnetic wave traveling in a homogeneous dielectric: 
$\vec{E} = (-100Vm^{-1})\hat{i}e^{i(kz-\omega t)}$. Here 
$\omega = 1.80 \times 10^{15}\frac{rad}{s}$ and $k = 1.20 \times 10^{7}\frac{rad}{m}$.\\
\noindent (a)Determine the associated $\vec{B}$-field.\\
\noindent $B= \frac{Em}{c} = \frac{-100Vm^{-1}}{3\times10^{8}ms^{1}} = -3.33\times10^{-7}T$\\\\
\noindent (b)Find the index of refraction\\
\noindent $v = f\lambda = 2\pi\frac{f\lambda}{2\pi} = \frac{\omega}{k} = \frac{1.8\times10^{15}s^{-1}rad}{1.2\times10^{7}m^{-1}} 
= 1.5\times10^{8}ms^{-1}$\\\\
\noindent $n = \frac{c}{v} = \frac{3\times10^{8}ms^{-1}}{1.5\times10^{8}ms^{-1}} = 2$\\\\
\noindent (c)Compute the permiativity\\
\noindent $\mu = \sqrt{(\frac{\mu_{0}\epsilon_{0}}{\mu_{0}})^{-1}} 
\rightarrow \epsilon^{\frac{1}{2}}=2\sqrt{8.85\times10^{-12}}
\rightarrow k = 3.5\times4\times10^{-12}$\\\\
\noindent (d)Find the irradiance\\
\noindent $I = \frac{C\epsilon_{0}E_{0}^{2}}{2} = =\frac{3\times10^{8}\times8.85\times10^{-12}\times(100)^{2}}{2} = 13.275Wm^{-2}$\\\\



\noindent 3.23) Consider a linearly polarized plane electromagnetic 
wave traveling in the positive x direction in free space having as its plane 
of vibration the xy-plane. Given that its frequency is 10MHz and its amplitude 
is $E_{0} = 0.08Vm^{-1}$, \\
(a) find the period of the wave\\
(b) write an expression for $E(t)$ and $B(t)$\\
(c) find the flux density, $\langle S \rangle$, of the wave.\\\\

\noindent (a) The period of an electromagnetic wave can be calculated as 
$\tau = \frac{1}{\nu} =\frac{1}{10^{7}s^{-1}} = 10^{-7}s.\\\\
\noindent (b) \lambda = \frac{v}{\nu} = \frac{c}{\nu} 
= \frac{3\times10^{8}ms^{-1}}{10^{-7}s^{-1}} = 30m.$\\\\
Since $E(x,t) = \hat{y}E_{0}\cos(k_{x}x-\omega t) 
= \hat{y}E_{0}\cos2\pi(\frac{x}{\lambda}-\nu t)
= \hat{y}(0.08Vm^{-1})\cos(2\pi(\frac{x}{30}-10^{7}t))$.\\\\

\noindent Recall that $\vec{B} = \frac{1}{c}\hat{k} \times \vec{E} 
=\frac{1}{c}\hat{x}\times\hat{y}\| \vec{E} \|$. Since the wave is 
propagating in the x direction, $\hat{k} = \hat{x}$. \\\\Thus
$\vec{B} = \hat{z}\frac{0.08Vm^{-1}}{2.998\times 10^{8}ms^{-1}}
\cos[2\pi(\frac{x}{30}-10^{7}t)] 
\\\\= \hat{z}(2.67\times10^{-10}Vsm^{-2})\cos[2\pi(\frac{x}{30}-10^{7}t)]
\\\\=\hat{z}(2.67\times10^{-10}T)\cos[2\pi(\frac{x}{30}-10^{7}t)]$.\\\\

\noindent (c) The flux density is given by $u = \epsilon_{0}E_{0}^{2}
=(8.825\times10^{-12}\frac{C^{2}}{Nm^{2}})(0.08Vm^{-1})^{2} = 5.648\times10^{-14}Jm^{-3}$.\\\\
\noindent The average magnitude of the Poynting vector is 
\\\\$\langle S \rangle = \frac{c\epsilon_{0}}{2}E_{0}^{2} 
= \frac{(3\times10^{8}ms^{1})(5.648\times10^{-14}Jm^{-3})}{2} 
= 8.472\times10^{-6}Wm^{-2}$.


\end{document}
