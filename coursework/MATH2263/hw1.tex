\documentclass[12pt]{article}
%\usepackage{tikz}
\usepackage{pgfplots}

\usepgfplotslibrary{fillbetween}

\usepgfplotslibrary{polar}



\usepackage{xfrac}
%\usepackage[demo]{graphicx}
\pgfplotsset{compat=1.11}
\usetikzlibrary{calc}

\usepackage{setspace}
\usepackage{extsizes}
\usepackage{pgfplots}
\usepackage{float}
\usepackage{amsmath, amsthm, amssymb}    
\usepackage[margin=.5in]{geometry}
\title{\vspace{-2.0cm}Homework 1}
\author{Joanne Wardell}
\date{Thursday, August 16, 2018}
\begin{document}
\maketitle

\subsection*{Section 11.1}
\noindent2.)$x=-\sqrt{t}$,\hspace{10pt}$y=t$,\hspace{10pt}$t \geq 0$\\\\
$x=-\sqrt{t}$\\
$-x=\sqrt{t}$\\
$(-x)^{2}=(\sqrt{t})^{2}$\\
$x^{2}=t$\\\\

\noindent$y = t = x^{2}$\\\\
The Cartesian equation is a parabola centered about the origin.
Enforcing the parameter boundary yeilds only the left side of the parabola.
The particle begins at the origin and moves rightward to $-\infty$ in the $x$ direction 
and $\infty$ in the $y$ direction.
The particle's starting point is the origin and the particle does not have a 
terminal point.\\
\begin{minipage}{\linewidth}
      \centering
      \begin{minipage}{0.45\linewidth}
          \begin{figure}[H]

\begin{tikzpicture}
\begin{axis}[
    xmin=-3,xmax=0,
    ymin=-1,ymax=8,
    grid=both,
    grid style={line width=.1pt, draw=gray!10},
    major grid style={line width=.2pt,draw=gray!50},
    axis lines=middle,
    minor tick num=5,
    enlargelimits={abs=0.5},
    axis line style={latex-latex},
    ticklabel style={font=\tiny,fill=white},
    xlabel style={at={(ticklabel* cs:1)},anchor=north west},
    ylabel style={at={(ticklabel* cs:1)},anchor=south west},
    restrict x to domain=-inf:0
]
	\addplot[mark=none, ultra thick]{x^2};

%\coordinate (O) at (0,0);
%\node[fill=white,circle,inner sep=0pt] (O-label) at ($(O)+(-135:10pt)$) {$O$};
\end{axis}
\end{tikzpicture}



          \end{figure}
      \end{minipage}
      \hspace{0.05\linewidth}
      \begin{minipage}{0.45\linewidth}
          \begin{figure}[H]



\noindent\begin{tabular}{c|c|c}
\textbf{t} & \textbf{x} & \textbf{y} \\ \hline
	0&0&0\\
	1&-1&1\\
	2&-$\sqrt{2}$&2\\
	3&-$\sqrt{3}$&3\\
	4&-$2$&4\\
	5&-$\sqrt{5}$&5\\
	6&-$\sqrt{6}$&6\\
\end{tabular}\\

          \end{figure}
      \end{minipage}
  \end{minipage}\pagebreak


\noindent2.)$x=4sin{t}$, $y=4cos{t}$, $0 \leq t \leq 2\pi$ \\\\
Squaring and adding both parametric equations lead the way to the Cartesian equation.\\\\
$(sin{t}=\frac{x}{4})^{2}$ $+$ $(cos{t}=\frac{y}{5})^2$\\\\
$=sin^{2}{t} + cos^{2}{t} = \frac{x^{2}}{4^{2}} + \frac{y^{2}}{5^{2}}$\\\\
$=1 = \frac{x^{2}}{16} + \frac{y^{2}}{25}$\\\\
$=1 = \frac{25x^{2}}{400} + \frac{16y^{2}}{25}$\\\\
$=1 = \frac{25x^{2} + 16y^{2}}{400}$\\\\
$=400 = \frac{25x^{2} + 16y^{2}}{400}$\\\\
$=1 = \frac{x^{2}}{16} + \frac{y^{2}}{25}$\\\\
The Cartesian equation is an ellipse centered at the origin with minor axis 10 and major axis 8. 
The particle begins at position $(0,5)$ and moves in a clock-wise direction. The terminal point for 
the particle is at position $(0,5)$, back at where it started.\\\\
\begin{minipage}{\linewidth}
      \centering
      \begin{minipage}{0.45\linewidth}
          \begin{figure}[H]

\begin{tikzpicture}
\begin{axis}[
    xmin=-6,xmax=6,
    ymin=-5,ymax=5,
    grid=both,
    grid style={line width=.1pt, draw=gray!10},
    major grid style={line width=.2pt,draw=gray!50},
    axis lines=middle,
    minor tick num=5,
    enlargelimits={abs=0.5},
    axis line style={latex-latex},
    ticklabel style={font=\tiny,fill=white},
    xlabel style={at={(ticklabel* cs:1)},anchor=north west},
    ylabel style={at={(ticklabel* cs:1)},anchor=south west},
    %restrict x to domain=-inf:0
]
\draw[black] (axis cs:0,0)ellipse [x radius=4, y radius=5];

%\coordinate (O) at (0,0);
%\node[fill=white,circle,inner sep=0pt] (O-label) at ($(O)+(-135:10pt)$) {$O$};
\end{axis}
\end{tikzpicture}


          \end{figure}
      \end{minipage}
      \hspace{0.05\linewidth}
      \begin{minipage}{0.45\linewidth}
          \begin{figure}[H]



\noindent\begin{tabular}{c|c|c}
\textbf{t} & \textbf{x} & \textbf{y} \\ \hline
	0		&0	&5\\
	$\frac{\pi}{2}$	&4	&0\\
	$\pi$		&0	&-5\\
	$\frac{3\pi}{2}$	&-4	&0\\
	$2\pi$		&0	&5\\
\end{tabular}

          \end{figure}
      \end{minipage}
  \end{minipage}\pagebreak



\noindent20.)$x = cos{t}$, \hspace{10pt} $y=2sin{t}$\\\\
The parametric equations trace the curve of an ellipse. \textbf{D.}\\\\\\
\noindent22.)$x = \sqrt{t}$, \hspace{10pt} $y=\sqrt{t}cos{t}$\\\\
Let's choose $t=0$. Then $x = \sqrt{t} = \sqrt{0} = 0$ and $y = \sqrt{t}cos{t} = (\sqrt{0})cos{0} = 0$.\\\\
So, at $t=0$, the parametric curve passes through the origin. That leaves us with choices A or E.\\\\
The graph of $\sqrt{x}$ is always positive, so the particle's $x$ position is never negative.\\\\
Thus, the correct choice is \textbf{A}.\pagebreak

\subsection*{Section 11.2}
\noindent4.)$x=cos{t}$, \hspace{10pt} $y=\sqrt{3}cos{t}$, \hspace{10pt} $t=\frac{2\pi}{3}$\\\\
Start by finding the slope of the tangent line at time $t=\frac{2\pi}{3}$.\\\\
$\frac{dy}{dt} = \frac{dy}{dx}\frac{dx}{dt}$\\\\
$\frac{dy}{dx} = \frac{\sfrac{dx}{dt}}{\sfrac{dy}{dt}}$\\\\
$\frac{\sfrac{dcos{t}}{dt}}{\sfrac{d\sqrt{3}cos{t}}{dt}}$\\\\
$\frac{-\sqrt{3}sin{t}}{-sin{t}} = \sqrt{3}$\\\\
Evaluate the parametric equations at time $t=\frac{2\pi}{3}$\\\\
$x=cos{\frac{2\pi}{3}} = -1$\\
$y=\sqrt{3}cos{\frac{2\pi}{3}} = -\sqrt{3}$\\\\

\noindent Assume the point slope form of a line.\\\\
$y-y_{1} = (x-x_{1})m$\\
$y+\sqrt{3} = (x+1)\sqrt{3}$\\
$y=\sqrt{3} + x\sqrt{3} -\sqrt{3}$\\
$y=  x\sqrt{3} $\\\\

\noindent Take the second derivative of the curve at time $t=\frac{3\pi}{2}$.\\\\
$\frac{d^{2}y}{dx^{2}} = \frac{dy'}{dx} = \frac{d\sqrt{3}}{dx} = 0$\pagebreak


\noindent26.)$x=t^{3}$, \hspace{10pt} $y=\frac{3t^{2}}{2}$, \hspace{10pt} $0 \leq t \leq \sqrt{3}$\\\\
$L=\int\limits_{a}^{b}\sqrt{(\frac{dx}{dt})^{2} + (\frac{dy}{dt})^{2}}dt$\\\\
$=\int\limits_{0}^{\sqrt{3}}\sqrt{(\frac{dx}{dt})^{2} + (\frac{dy}{dt})^{2}}dt$\\\\
$=\int\limits_{0}^{\sqrt{3}}\sqrt{(3t^{2})^{2} + (3t)^{2}}dt$\\\\
$=\int\limits_{0}^{\sqrt{3}}\sqrt{9t^{4} + 9t^{2}}dt$\\\\
$=3\int\limits_{0}^{\sqrt{3}}t\sqrt{t^{2} + 1}dt$\\\\
Let $u=t^{2}+1$. Then $du = 2t$ $dt$ and $dt = \frac{du}{2t}$. The bounds of integration are adjusted. \\\\
$=\frac{3}{2}\int\limits_{1}^{4}\sqrt{u}du$
$=\frac{3}{2}\frac{u^{\frac{3}{2}}}{\sfrac{3}{2}}\Big|_1^4$
$=u^{\frac{3}{2}}\Big|_1^4$
$=(\sqrt{4})^{3} - (\sqrt{1})^{2} =$ \textbf{7}\\\pagebreak

\subsection*{Section 11.3}
\noindent4.)




\pgfplotsset{mypolarplot/.style={%
  clip=false, % needed for double line (last \addplot command)
  domain=0:360, % plot full cycle
  samples=180, % number of samples; can be locally adjusted
  grid=both, % display major and minor grids
  major grid style={black},
  minor x tick num=3, % 3 minor x ticks between majors
  minor y tick num=1, % 1 minor y tick between majors
  xtick={0,45,...,359},
  xticklabels={%
    $0$,
    $\frac{ \pi}{4}$,
    $\frac{ \pi}{2}$,
    $\frac{3\pi}{4}$,
    $\pi$,
    $\frac{5\pi}{4}$,
    $\frac{3\pi}{2}$,
    $\frac{7\pi}{4}$
  },
  yticklabel style={anchor=north}, % move label position
}}

\begin{tikzpicture}
\begin{polaraxis}[%
  ymax=4,
  ytick={0,1,2,3},
	  mypolarplot
]
%\addplot[mark=*] coordinates{ (45:3) (45:-3) (-45:3) (-45:-3) };
	\node[circle,draw=black, fill=black, inner sep=0pt,minimum size=5pt] at (45:2.58cm){};
	\node[circle,draw=black, fill=black, inner sep=0pt,minimum size=5pt] at (45:-2.58cm){};
	\node[circle,draw=black, fill=black, inner sep=0pt,minimum size=5pt] at (-45:2.58cm){};
	\node[circle,draw=black, fill=black, inner sep=0pt,minimum size=5pt] at (-45:-2.58cm){};
%\pgfpathcircle{\pgfpointpolar{45}{4.25}}{2pt}
%\tikz \draw   (45:3) (45:-3) (-45:3) (-45:-3); 
\end{polaraxis}
\end{tikzpicture}\\

\noindent(a.)polar coordinate $(3, \frac{\pi}{4})$\\\\
If $(r, \theta)$ is the same point as $(-r, \theta \pm \pi)$, then $(-3, \frac{5\pi}{4})$, $(-3, -\frac{3\pi}{4})$, and $(3, \frac{\pi}{4})$ are the same point. \\
When $r=-3$, the polar coordinates are of the form $(-3, \frac{5\pi}{4}+ 2\pi n)$ where $n$ is an integer.\\
When $r=3$, the polar coordinates are of the form $(3,\frac{\pi}{4}+ 2\pi n)$ where $n$ is an integer.\\\\


\noindent(b.)polar coordinate $(-3, \frac{\pi}{4})$\\\\
The point is equally described using polar coordinates $(3, \frac{5\pi}{4})$ and $(3, -\frac{3\pi}{4})$.\\
When $r=-3$, the polar coordinates are of the form $(-3, \frac{\pi}{4}+ 2\pi n)$ where $n$ is an integer.\\
When $r=3$, the polar coordinates are of the form $(3,\frac{5\pi}{4}+ 2\pi n)$ where $n$ is an integer.\\\\



\noindent(c.)polar coordinate $(3, -\frac{\pi}{4})$\\\\
The point is equally described using polar coordinates $(-3, \frac{3\pi}{4})$ and $(-3, -\frac{5\pi}{4})$.\\
When $r=-3$, the polar coordinates are of the form $(-3, \frac{3\pi}{4}+ 2\pi n)$ where $n$ is an integer.\\
When $r=3$, the polar coordinates are of the form $(3,-\frac{\pi}{4} + 2\pi n)$ where $n$ is an integer.\\\\



\noindent(d.)polar coordinate $(-3, -\frac{\pi}{4})$\\\\
The point is equally described using polar coordinates $(3, \frac{3\pi}{4})$ and $(3, -\frac{5\pi}{4})$.\\
When $r=-3$, the polar coordinates are of the form $(-3, -\frac{\pi}{4}+ 2\pi n)$ where $n$ is an integer.\\
When $r=3$, the polar coordinates are of the form $(3,\frac{3\pi}{4}+ 2\pi n)$ where $n$ is an integer.\\\\


\pagebreak

\noindent12.)The set of points that satisfy $0 \leq r \leq 2$ are those that lie within circles of radii $0$ to $2$ inclusive.
This also includes the points on the exterior of the disk of radius $2$.\\
\pgfplotsset{mypolarplot/.style={%
  clip=false, % needed for double line (last \addplot command)
  domain=0:360, % plot full cycle
  samples=180, % number of samples; can be locally adjusted
  grid=both, % display major and minor grids
  major grid style={black}, 
  minor x tick num=3, % 3 minor x ticks between majors
  minor y tick num=1, % 1 minor y tick between majors
  xtick={0,45,...,359},
  xticklabels={%
    $0$,
    $\frac{ \pi}{4}$,
    $\frac{ \pi}{2}$,
    $\frac{3\pi}{4}$,
    $\pi$,
    $\frac{5\pi}{4}$,
    $\frac{3\pi}{2}$,
    $\frac{7\pi}{4}$
  },
  yticklabel style={anchor=north}, % move label position
}}

\begin{tikzpicture}
\begin{polaraxis}[%
  ymax=4,
  ytick={0,1,2,3},
  mypolarplot,
]
	\addplot[mark=none,fill=red!70!black,opacity=0.5,domain=0:360] {2};
	\addplot[mark=none,thick,red!70!black] {2};
  \addplot[black] {4.05}; % there is likely a better way to do this
\end{polaraxis}
\end{tikzpicture}\\\\






\noindent14.)The set of points that satisfy $1 \leq r \leq 2$ are those that lie within circles of radii $1$ to $2$ inclusive.
Since points $1 < r <  2$ are omitted, the points are enclosed in a washer.
The points on the interior and exterior edges of the washer also satisfy the inequality.\\
\pgfplotsset{mypolarplot/.style={%
  clip=false, % needed for double line (last \addplot command)
  domain=0:360, % plot full cycle
  samples=180, % number of samples; can be locally adjusted
  grid=both, % display major and minor grids
  major grid style={black}, 
  minor x tick num=3, % 3 minor x ticks between majors
  minor y tick num=1, % 1 minor y tick between majors
  xtick={0,45,...,359},
  xticklabels={%
    $0$,
    $\frac{ \pi}{4}$,
    $\frac{ \pi}{2}$,
    $\frac{3\pi}{4}$,
    $\pi$,
    $\frac{5\pi}{4}$,
    $\frac{3\pi}{2}$,
    $\frac{7\pi}{4}$
  },
  yticklabel style={anchor=north}, % move label position
}}

\begin{tikzpicture}
\begin{polaraxis}[%
  ymax=4,
  ytick={0,1,2,3},
  mypolarplot
]
	\addplot[name path=outer,mark=none,thick,red!70!black] {2};
	\addplot[name path=inner,mark=none,thick,red!70!black] {1};
	%\addplot[mark=none,fill=red!70!black,opacity=0.5] fill between[of=f and g,soft clip={domain=1:4}] {2};
	\pgfsetlayers{axis background}
	\pgfdeclarelayer{axis background}
	    \begin{pgfonlayer}{axis background}
		    \fill[red!70!black,opacity=0.5, intersection segments={of=inner and outer}];
	    \end{pgfonlayer}



  \addplot[black] {4.05}; % there is likely a better way to do this
\end{polaraxis}
\end{tikzpicture}



\pagebreak

\noindent18.)$\theta=\frac{11\pi}{4}$, \hspace{10pt} $r \leq -1$\\\\
The points that satisfy this inequality exist on a ray with starting point one unit away from the 
origin. The ray is oriented at an angle of $\frac{3\pi}{4}$ from the positive $x$ axis and its tip extends 
away from the origin.\\
\pgfplotsset{mypolarplot/.style={%
  clip=false, % needed for double line (last \addplot command)
  domain=0:360, % plot full cycle
  samples=180, % number of samples; can be locally adjusted
  grid=both, % display major and minor grids
  major grid style={black}, 
  minor x tick num=3, % 3 minor x ticks between majors
  minor y tick num=1, % 1 minor y tick between majors
  xtick={0,45,...,359},
  xticklabels={%
    $0$,
    $\frac{ \pi}{4}$,
    $\frac{ \pi}{2}$,
    $\frac{3\pi}{4}$,
    $\pi$,
    $\frac{5\pi}{4}$,
    $\frac{3\pi}{2}$,
    $\frac{7\pi}{4}$
  },
  yticklabel style={anchor=north}, % move label position
}}

\begin{tikzpicture}
\begin{polaraxis}[%
  ymax=4,
  ytick={0,1,2,3},
  mypolarplot,
]
  \addplot[black] {4.05}; % there is likely a better way to do this
	\node[circle,draw=red, fill=red, inner sep=0pt,minimum size=5pt] at (135:-.85cm){};
	\draw[red,very thick,->] (135:-.85cm) -- (135:3.5cm);
\end{polaraxis}
\end{tikzpicture}

\noindent24.)$-\frac{\pi}{4} \leq \theta \leq \frac{\pi}{4}$, \hspace{10pt} $-1 \leq r \leq 1$\\\\
The points that satisfy the inequalities lie within a washer segment partially in quadrants $I$ and $IV$.\\
\pgfplotsset{mypolarplot/.style={%
  clip=false, % needed for double line (last \addplot command)
  domain=0:360, % plot full cycle
  samples=180, % number of samples; can be locally adjusted
  grid=both, % display major and minor grids
  major grid style={black}, 
  minor x tick num=3, % 3 minor x ticks between majors
  minor y tick num=1, % 1 minor y tick between majors
  xtick={0,45,...,359},
  xticklabels={%
    $0$,
    $\frac{ \pi}{4}$,
    $\frac{ \pi}{2}$,
    $\frac{3\pi}{4}$,
    $\pi$,
    $\frac{5\pi}{4}$,
    $\frac{3\pi}{2}$,
    $\frac{7\pi}{4}$
  },
  yticklabel style={anchor=north}, % move label position
}}

\begin{tikzpicture}
\begin{polaraxis}[%
  ymax=4,
  ytick={0,1,2,3},
  mypolarplot,
]
  \addplot[black] {4.05}; % there is likely a better way to do this
  \addplot[mark=none,fill=red!70!black,opacity=0.5,domain=-45:45] {1};
  \addplot[mark=none,fill=blue!70!black,opacity=0.5,domain=-45:45] {-1};

\end{polaraxis}
\end{tikzpicture}



\noindent32.) $r=-3sec{\theta}$\\\\
$(r=-3sec{\theta})\frac{1}{sec{\theta}}$\\
$rcos{\theta} = -3$\\
$x=-3$\\\\

\noindent The Cartesian equation $x=-3$ is a vertical line that passes through the point $(-3,0)$.\\\\\\


\subsection*{Section 11.6}

\noindent56.)$x-y=3$\\\\
$x - y = 3$\\\\
$rcos{\theta} - rsin{\theta} = 3$\\\\
$r(cos{\theta} - sin{\theta}) = 3$\\\\
$r = \frac{3}{cos{\theta} - sin{\theta}}$\\\\\\

\noindent58.)$2x^{2}+2y^{2}-28x+12y+114 = 0$\\\\
$2x^{2}-28x+2y^{2}+12y+114=0$\\
$x^{2}-14x+y^{2}+6y+57=0$\\
$x^{2}-14x+48+y^{2}+6y+9=0$\\
$x^{2}-14x+49+y^{2}+6y+9=1$\\\\
$(x-7)^{2} + (y+3)^{2} = 1$\\\\
The conic section is a circle centered about $(7,3)$ with radius $1$.\\\\



\noindent66.)$x^{2}-y^{2}+4x-6y = 6$\\\\
$x^{2}+4x-y^{2}-6y=6$\\
$x^{2}+4x+4-y^{2}-6y-9=6+4-9$\\\\
$(x+2)^{2}-(y+3)^{2}=1$\\

\noindent The conic section is a hyperbola with foci on the x-axis. It is of the form $\frac{x-h}{a^{2}} - \frac{y-k}{b^{2}} = 1$.\\\\
\textbf{center:} $(h, k) = (-2, -3)$\\
\textbf{vertices:}$(h\pm a, k)=(-1,-3)$ and $(-3,-3)$\\
\textbf{center to focus distance:} $c=\sqrt{a^{2}+b^{2}} = \sqrt{2}$\\
\textbf{foci:} $(h\pm c, k) = (-2+\sqrt{2}, -3)$ and $(-2-\sqrt{2},-3)$\\
\textbf{assymptotes:} $y-k=\frac{b}{a}(x-h)$\hspace{10pt} $y = -x -5$ and $y = x-1$












\end{document}
