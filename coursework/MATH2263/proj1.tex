\documentclass[12pt]{article}
\usepackage{tikz}
\usepackage{harpoon}
\usepackage{setspace}
\usepackage{extsizes}
\usepackage{float}
\usepackage{amsmath, amsthm, amssymb}    
\usepackage[margin=.5in]{geometry}
\usepackage{pgfplots}


\title{\vspace{-2.0cm}Project 1}
\author{Joanne Wardell}
\date{Thursday, September 13, 2018}
\begin{document}
\maketitle
\noindent 1.) $\frac{d\vec{r}}{dt} = \vec{v}(t) = \langle cos(t) -tsin(t), sin(t) + tcos(t), 1\rangle$\\\\
\noindent 2.) $\| \vec{v}\|  = \sqrt{(cos(t) - tsin(t))^{2} + (sin(t) + tcos(t)) + 1^{2}}
\\\\= \sqrt{cos^{2}(t) + 2tcos(t)sin(t) + t^{2}sin^{2} + sin^{2}(t) -2tcos(t)sin(t) + t^{2}cos^{2}(t) + 1} 
\\\\ = \sqrt{t^{2} + 2}$\\\\
\noindent the plot of speed\\
\begin{tikzpicture}
  \begin{axis}[
      xmin=1,xmax=3,
      ymin=-1,ymax=8,
      grid=both,
      grid style={line width=.1pt, draw=gray!10},
      major grid style={line width=.2pt,draw=gray!50},
      axis lines=middle,
      minor tick num=5,
      enlargelimits={abs=0.5},
      axis line style={latex-latex},
      ticklabel style={font=\tiny,fill=white},
      xlabel style={at={(ticklabel* cs:1)},anchor=north west},
      ylabel style={at={(ticklabel* cs:1)},anchor=south west},
      restrict x to domain=-1:inf
  ]
	  \addplot[mark=none, ultra thick]{sqrt(x^2 + 2)};
 
\end{axis}
\end{tikzpicture}



\noindent As $t \in [0, \infty)$ varies, the speed function increases and is never negative.
A particle in the tornado moves faster as time progresses.\\\\
\noindent The unit tangent vector $\vec{T}$ is the normalized velocity vector:\\\\
$\frac{\vec{v}}{\| \vec{v} \|} = \frac{1}{\sqrt{t^{2}+ 2}}\langle cos(t) -tsin(t), sin(t) + tcos(t), 1\rangle$\\\\
\noindent 3.) $\vec{a} = \frac{d\vec{v}}{dt} = \langle - 2sin(t) -tcos(t), 2cos(t) - tsin(t), 0\rangle$\\\\
\noindent $\| \vec{a} \| = \sqrt{(-2sin(t)-tcos(t))^{2} + (2cos(t) - tsin(t))^{2} + 0^{2}} \\\\= 
\sqrt{4sin^{2}(t)+4tcos(t)sin(t)+t^{2}cos^{2}(t) + 4cos^{2}(t)-4tcos(t)sin(t)+t^{2}sin^{2}(t)} \\\\= \sqrt{t^{2} + 4}$\clearpage


\noindent the plot of acceleration's magnitude\\
\begin{tikzpicture}
  \begin{axis}[
      xmin=1,xmax=3,
      ymin=-1,ymax=8,
      grid=both,
      grid style={line width=.1pt, draw=gray!10},
      major grid style={line width=.2pt,draw=gray!50},
      axis lines=middle,
      minor tick num=5,
      enlargelimits={abs=0.5},
      axis line style={latex-latex},
      ticklabel style={font=\tiny,fill=white},
      xlabel style={at={(ticklabel* cs:1)},anchor=north west},
      ylabel style={at={(ticklabel* cs:1)},anchor=south west},
      restrict x to domain=-1:inf
  ]
	  \addplot[mark=none, ultra thick]{sqrt(x^2 + 4)};
 
\end{axis}
\end{tikzpicture}

\noindent From the previous example, we noticed that the speed was not constant but always 
increasing. It was increasing non-linearly as well, so the acceleration is expected to be variable, that is, 
not a constant as time progresses. As expected, the acceleration is not constant and it's magnitude is also 
not constant. In fact, acceleration's magnitude is increasing. All of the remaining derivatives will be non-constant 
because of the sqaure root in the functions.\\\\


\noindent 4.) $a_{T} = \frac{d}{dt}[\| \vec{v} \|] = \frac{d}{dt}[\sqrt{t^{2} + 2}] = \frac{t}{\sqrt{t^{2} + 2}}$\\\\
\noindent as $t \to \infty$, $\frac{t}{t^{2} + 2} \approx 1$\\\\
\noindent $a_{N} = \sqrt{\| a \| ^{2} + a_{T}^{2}}\\ 
\\= \sqrt{(\sqrt{t^{2} + 4})^{2} + (\frac{1}{\sqrt{t^{2} + 2}})^{2}}\\
\\= \sqrt{\frac{(t^{2} + 4)(t^{2} + 2)}{(t^{2} + 2)} + \frac{1}{t^{2} + 2}}\\
\\ = \sqrt{\frac{t^{4} + 6t^{2} + 9}{t^{2} + 2}}\\\\ $ as $t \to \infty$, $\sqrt{\frac{(t^{2} + 4)(t^{2} + 2)}{(t^{2} + 2)} + \frac{1}{t^{2} + 2}}\approx \sqrt{\frac{t^{4}}{t^{2}}} = t $\\\\
\noindent graphs of the tangential and normal components\\\\
\begin{tikzpicture}
  \begin{axis}[
      xmin=1,xmax=3,
      ymin=-1,ymax=8,
      grid=both,
      grid style={line width=.1pt, draw=gray!10},
      major grid style={line width=.2pt,draw=gray!50},
      axis lines=middle,
      minor tick num=5,
      enlargelimits={abs=0.5},
      axis line style={latex-latex},
      ticklabel style={font=\tiny,fill=white},
      xlabel style={at={(ticklabel* cs:1)},anchor=north west},
      ylabel style={at={(ticklabel* cs:1)},anchor=south west},
      restrict x to domain=-1:inf
  ]
	  \addplot[mark=none, ultra thick, label=woo1]{sqrt((x^4 + 6*x^2 + 9)/(x^2 + 2))};
	  \addplot[mark=none, ultra thick, label=woo2]{sqrt(1/(x^2 + 2))};
 
\end{axis}
\end{tikzpicture}


\noindent 5.) $\kappa = \frac{a_{N}}{\| \vec{v} \| ^{2}} = \frac{}{} = $\\\\
\noindent 6.)  $\vec{v} \times \vec{a} = $\\\\
\noindent $\| \vec{v} \times \vec{a} \| = $

\noindent 7.)$\tau = \begin{vmatrix}
      x    &  y   &z \\ 
      x''  &  y'' &z'' \\
      x''' &  y'''&z''' \\ 
\end{vmatrix}
\frac{1}{\| \vec{v} \times \vec{a}\|^{2}}$
\noindent plot of $\tau$\\
\noindent explaination of $\tau$'s plot
\end{document}
